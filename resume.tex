%----------------------------------------------------------------------------------------
%  PAQUETS ET CONFIGURATIONS
%----------------------------------------------------------------------------------------

\documentclass[10pt]{style}

\usepackage[T1]{fontenc}
\usepackage[utf8]{inputenc}
\usepackage{ebgaramond}
\usepackage{hyperref}
\usepackage{graphicx}
\usepackage{microtype} % meilleure césure, gain de lignes

% -- serrer un peu les marges par-dessus la classe (juste ce qu'il faut)
\usepackage{geometry}
\geometry{
  top=0.55in, bottom=0.55in,
  left=0.70in, right=0.70in
}

% -- tableaux plus compacts (compétences)
\setlength{\tabcolsep}{4pt}
\renewcommand{\arraystretch}{0.95}

\begin{document}

%----------------------------------------------------------------------------------------
%  EN-TÊTE : PHOTO À GAUCHE, TEXTE CENTRÉ DANS LE BLOC DE DROITE (compact)
%----------------------------------------------------------------------------------------
\begin{center}
  \begin{minipage}[c]{0.18\textwidth}
    \includegraphics[width=\linewidth,height=30mm,keepaspectratio]{photo.jpg}% 30mm au lieu de 34mm
  \end{minipage}\hfill
  \begin{minipage}[c]{0.78\textwidth}
    \centering
    {\LARGE\bfseries Hénoc MUKUMBI}\\[-0.2em]
    \small (+33) 7\,67\,17\,82\,93 \quad\textbullet\quad \texttt{h.mukumbi100@gmail.com} \quad\textbullet\quad
    119 avenue du Général Leclerc, 78220 Viroflay\\[-0.1em]
    \normalsize \textbf{Étudiant en Systèmes Embarqués à l’ESIEE Paris — Recherche apprentissage 3 ans (Diplôme 2028).}
  \end{minipage}
\end{center}

% \vspace{0.8em}  % <-- supprimé pour gagner de l’espace

%----------------------------------------------------------------------------------------
%  EXPÉRIENCE PROFESSIONNELLE
%----------------------------------------------------------------------------------------

\begin{rSection}{Expérience}

  \begin{rSubsection}{LGM Ingénierie}{Août 2023 -- Août 2025}{Apprenti Développeur en Systèmes Embarqués}{Vélizy-Villacoublay}
    \item Validation et tests couche physique sur calculateurs automobile (CAN, LIN) avec CANoe, oscilloscopes et modules de perturbation.
    \item Intégration et débogage de logiciels embarqués (Eclipse, IBM Rational Test RealTime).
    \item Développement et exécution de tests unitaires et sur cible en C pour des composants critiques en temps réel.
    \item Utilisation de Fork pour la gestion de versions et le développement collaboratif en environnements de validation.
    \item Interface web interne (Python/Django/SQL) pour automatiser les rapports, réduire le temps d’analyse et améliorer la traçabilité.
    \item Collaboration interdisciplinaire (logiciel, validation, qualité) pour atteindre les jalons et garantir la conformité.
  \end{rSubsection}

  \begin{rSubsection}{CNRS -- Université Paris Cité}{Avril 2023 -- Juin 2023}{Stagiaire Technicien Systèmes Embarqués et Électronique}{Paris}
    \item Conception et routage de circuits imprimés (Ultiboard) pour instrumentation (contrôle de température, salle blanche).
    \item Assemblage, câblage et débogage de cartes ; intégration et calibration de capteurs de température.
    \item Développement et migration de firmware embarqué en C/C++ sur PIC18F26J11 pour la régulation en temps réel.
    \item Réalisation et test d’une liaison RF avec modules Si4432.
  \end{rSubsection}

\end{rSection}

%----------------------------------------------------------------------------------------
%  FORMATION
%----------------------------------------------------------------------------------------

\begin{rSection}{Formation}

  \textbf{ESIEE Paris} \hfill \textit{2025--2028} \\
  Diplôme d’ingénieur — Systèmes Embarqués

  \textbf{Université de Technologie de Compiègne (UTC)} \hfill \textit{2023--2025} \\
  Licence en Informatique

  \textbf{Université Paris-Saclay — IUT de Cachan} \hfill \textit{2021--2023} \\
  DUT Génie Électrique et Informatique Industrielle

\end{rSection}

%----------------------------------------------------------------------------------------
%  COMPÉTENCES TECHNIQUES
%----------------------------------------------------------------------------------------

\begin{rSection}{Compétences Techniques}

  \begin{tabular}{@{} >{\bfseries}l @{\hspace{4ex}} l @{}}
    Langages & C, C++, Python, VHDL, JavaScript, HTML/CSS, SQL \\
    Embarqué & ARM Mbed, Arduino, Raspberry Pi, PIC \\
    Protocoles & SPI, I2C, CAN, UART, Ethernet \\
    PCB \& CAO & KiCad, Ultiboard, Altium, SolidWorks \\
    Réseaux & TCP/IP, routage, firewall, LAN/WAN, sockets \\
    BD \& OS & PostgreSQL, SQLite, Windows, macOS, Ubuntu \\
    Outils & Git, GitHub, GitLab, Eclipse, LabVIEW, Matlab \\
  \end{tabular}

\end{rSection}

%----------------------------------------------------------------------------------------
%  PROJETS PERSONNELS
%----------------------------------------------------------------------------------------

\begin{rSection}{Réalisations}

  \begin{rSubsection}{Drone aérien personnalisé}{2025 - Présent}{}{}
    \item Étude et analyse des schémas électroniques (alimentation, étage de puissance, communication) pour garantir la fiabilité et la sécurité du système.
    \item Conception mécanique (SolidWorks, impression 3D) et intégration structurelle optimisée.
    \item Conception électronique complète : ESC basé sur TI F280025PMS, carte de contrôle de vol conçue sous KiCad.
    \item Intégration multi-capteurs (IMU, capteurs environnementaux, capteur optique) et développement logiciel embarqué en C++ (protocoles SPI, I2C, UART).
    \item Validation par simulations et tests sur prototype pour assurer robustesse et performances en conditions réelles.
  \end{rSubsection}

\end{rSection}



%----------------------------------------------------------------------------------------
%  LANGUES
%----------------------------------------------------------------------------------------

\begin{rSection}{Langues}
  Français — Langue maternelle \quad|\quad Anglais — Niveau professionnel
\end{rSection}

\enlargethispage{\baselineskip}

\end{document}
