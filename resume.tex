%----------------------------------------------------------------------------------------
%  PACKAGES & SETTINGS
%----------------------------------------------------------------------------------------

\documentclass[10pt]{style}

\usepackage[T1]{fontenc}
\usepackage[utf8]{inputenc}
\usepackage{ebgaramond}
\usepackage{hyperref}
\usepackage{graphicx}
\usepackage[protrusion=true,expansion=false]{microtype}
\usepackage{geometry}

\setlength{\tabcolsep}{3.5pt}
\renewcommand{\arraystretch}{0.9}

\begin{document}

%----------------------------------------------------------------------------------------
%  HEADER
%----------------------------------------------------------------------------------------
\begin{center}
  \begin{minipage}[c]{0.20\textwidth}
    \includegraphics[width=\linewidth,height=26mm,keepaspectratio]{photo.jpg}
  \end{minipage}\hfill
  \begin{minipage}[c]{0.76\textwidth}
    \centering
    {\LARGE\bfseries Hénoc MUKUMBI}\\[-0.2em]
    \small (+33) 7\,67\,17\,82\,93 \quad\textbullet\quad \texttt{h.mukumbi100@gmail.com} \quad\textbullet\quad
    119 avenue du Général Leclerc, 78220 Viroflay (France)\\[-0.1em]
    \normalsize \textbf{Candidature — Alternant Ingénieur Bureau d’Études en Industrie}\newline
    \small Français — Langue maternelle \quad|\quad Anglais — Niveau professionnel
  \end{minipage}
\end{center}

%----------------------------------------------------------------------------------------
%  MOTS-CLÉS ATS
%----------------------------------------------------------------------------------------
\begin{center}
  \textit{Mots-clés : Bureau d’études • Conception électronique • Innovation • Industrialisation • Méthodes • CAO électronique et mécanique • Tests et validation • Amélioration continue • Documentation technique • Fiabilité produit}
\end{center}

%----------------------------------------------------------------------------------------
%  EXPERIENCE
%----------------------------------------------------------------------------------------
\begin{rSection}{Expérience}

  \begin{rSubsection}{LGM Ingénierie}{Août 2023 -- Août 2025}{Apprenti Ingénieur Systèmes Embarqués}{Vélizy-Villacoublay}
    \item Études et validation de calculateurs automobiles (CAN, LIN) avec CANoe, oscilloscopes et multimètres.  
    \item Développement et exécution de tests unitaires/sur cible en C avec IBM Rational Test RealTime.  
    \item Participation à l’amélioration des méthodes de validation et de traçabilité qualité.  
    \item Collaboration transverse (logiciel, validation, qualité) en environnement industriel.  
  \end{rSubsection}

  \begin{rSubsection}{CNRS — Université Paris Cité}{Avril 2023 -- Juin 2023}{Stagiaire Technicien Bureau d’Études Électronique}{Paris}
    \item Conception et routage de circuits imprimés (Ultiboard) pour instrumentation scientifique.  
    \item Assemblage, réparation, calibration et intégration de cartes électroniques.  
    \item Réalisation de tests fonctionnels et contribution à la rédaction de documentation technique.  
  \end{rSubsection}

\end{rSection}

%----------------------------------------------------------------------------------------
%  EDUCATION
%----------------------------------------------------------------------------------------
\begin{rSection}{Formation}

  \textbf{ESIEE Paris} \hfill \textit{2025--2028} \\
  Diplôme d’ingénieur — Systèmes Embarqués (en alternance)  

  \textbf{Université de Technologie de Compiègne (UTC)} \hfill \textit{2023--2025} \\
  Licence en Informatique  

  \textbf{Université Paris-Saclay — IUT de Cachan} \hfill \textit{2021--2023} \\
  DUT Génie Électrique et Informatique Industrielle  

\end{rSection}

%----------------------------------------------------------------------------------------
%  SKILLS
%----------------------------------------------------------------------------------------
\begin{rSection}{Compétences Techniques}

  \begin{tabular}{@{} >{\bfseries}l @{\hspace{3.5ex}} l @{}}
    Bureau d’études & Analyse de besoins, conception schémas, documentation technique \\
    Conception PCB & KiCad, Ultiboard, Altium Designer \\
    Méthodes & Rédaction modes opératoires, amélioration continue, Lean (bases) \\
    Tests & Oscilloscope, GBF, multimètre, analyseur logique \\
    Développement & C, C++, Python, SQL, VHDL \\
    CAO mécanique & SolidWorks, impression 3D \\
    Outils & GitHub, GitLab, Eclipse, Matlab, Pack Office \\
  \end{tabular}

\end{rSection}

%----------------------------------------------------------------------------------------
%  PROJECTS
%----------------------------------------------------------------------------------------
\begin{rSection}{Réalisations}

  \begin{rSubsection}{Drone aérien personnalisé}{2025 -- Présent}{Projet personnel}{}
    \item Étude des schémas électroniques (alimentation, puissance, communication).  
    \item Développement d’une carte de contrôle moteurs (TI F280025PMS) et intégration capteurs.  
    \item Validation sur prototype par mesures et bancs de tests.  
  \end{rSubsection}

  \begin{rSubsection}{Capteur de luminosité}{Sept. 2025 -- Présent}{Projet académique}{ESIEE Paris}
    \item Étude des schémas et dimensionnement des composants (résistances, condensateurs).  
    \item Début du routage PCB sous KiCad et mise en place de la documentation technique.  
  \end{rSubsection}

\end{rSection}

\end{document}