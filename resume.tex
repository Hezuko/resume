%----------------------------------------------------------------------------------------
%	PACKAGES AND OTHER DOCUMENT CONFIGURATIONS
%----------------------------------------------------------------------------------------

\documentclass[
	10pt,
]{style} % Use the style class

\usepackage{ebgaramond}
\usepackage{hyperref} % Use the EB Garamond font

%------------------------------------------------

\name{Hénoc MUKUMBI}

\address{(+33) 7~$\cdot$~67~$\cdot$~17~$\cdot$~82~$\cdot$~93 \\ \texttt{h.mukumbi100@gmail.com}} % Contact information

%----------------------------------------------------------------------------------------

\begin{document}

\begin{center}
	\url{https://www.linkedin.com/in/henocmukumbi/}
\end{center}

\begin{center}
	\textbf{Graduating student in 2026 from Université de Technologie de Compiegne.}
\end{center}

%----------------------------------------------------------------------------------------
%	WORK EXPERIENCE SECTION
%----------------------------------------------------------------------------------------

\begin{rSection}{Experience}

	\begin{rSubsection}{LGM Ingénierie}{January 2025 - Present}{Apprenticeship - Full-Stack Web Developer}{Vélizy-Villacoublay, France}
		\item Designed and developed a web application to generate validation reports for automotive Electronic Control Units (ECUs), replacing a previous third-party solution. The application was built using Python (Django), HTML, CSS (Bootstrap), JavaScript, and HTMX.
		\item Collaborated with the software development team, working closely with the validation service lead to define weekly objectives and align with project specifications.
	\end{rSubsection}

%------------------------------------------------

	\begin{rSubsection}{LGM Ingénierie}{August 2024 - January 2025}{Apprenticeship - Automotive Validation Engineer}{Vélizy-Villacoublay, France}
		\item Conducted validation testing on automotive ECUs, focusing on lifecycle phases and physical layers based on Stellantis requirements.
		\item Developed expertise in CAN and LIN protocols, analyzing their applications in vehicle communication networks.
		\item Utilized tools such as CANoe, oscilloscopes, multimeters, and CAN disturbance modules to assess ECU performance and compliance.
		\item Improved test planning and execution efficiency, optimizing workflow to meet project deadlines.
	\end{rSubsection}


%------------------------------------------------

	\begin{rSubsection}{LGM Ingénierie}{August 2023 - July 2024}{Apprenticeship - Software Test Engineer}{Vélizy-Villacoublay, France}
		\item Performed unit testing on OCEAN software modules, a communication framework ensuring reliable data exchange between automotive ECUs.
		\item Conducted on-target unit tests to validate software functionality in real-world conditions.
		\item Reviewed code compliance with LGM Ingénierie standards ensuring quality and robustness.
	\end{rSubsection}

%------------------------------------------------

	\begin{rSubsection}{CNRS - Université Paris Cité}{April 2023 - June 2023}{Internship - Embedded Electronics and Systems Technician}{Paris, France}
		\item Designed and debugged PCBs using Ultiboard integrating temperature sensors for clean room monitoring.
		\item Implemented a Si4432 RF for wireless data transmission.
		\item Migrated firmware from ATmega328P to PIC18F26J11 optimizing the temperature regulation system in C/C++.
	\end{rSubsection}

\end{rSection}

%----------------------------------------------------------------------------------------
%	EDUCATION SECTION
%----------------------------------------------------------------------------------------

\begin{rSection}{Education}

	\textbf{Université de Technologie de Compiègne (UTC) - Sorbonne} \hfill \textit{June 2026} \\
	Master in Computer Science \\
	Specializing in Embedded Computing and Autonomous Systems

	\textbf{Université Paris-Saclay IUT de Cachan} \hfill \textit{June 2023} \\
	Associate's Degree in Electrical Engineering and Industrial Computing \\
	Average : 15.0/20

\end{rSection}

%----------------------------------------------------------------------------------------
%	TECHNICAL STRENGTHS SECTION
%----------------------------------------------------------------------------------------

\begin{rSection}{Technical Strengths}

	\begin{tabular}{@{} >{\bfseries}l @{\hspace{6ex}} l @{}}
		Computer Languages & C, C++, Python, VHDL, JavaScript, HTML, CSS \\
		Embedded Systems & ARM Mbed, Arduino, Raspberry Pi, Microchip PIC \\
		Communication Protocols & SPI, I2C, Ethernet, CAN Bus, LIN Bus, RS232, UART \\
		PCB Design \& CAD & KiCad, Ultiboard, Altium Designer, SolidWorks \\
		Web Technologies & Node.js, Express.js, Django, Flask \\
		Databases & PostgreSQL, SQLite \\
		Operating Systems & Windows, macOS, Ubuntu \\
		Workflow \& Tools & Git, GitHub, GitLab \\
	\end{tabular}
	

\end{rSection}

\end{document}
